\documentclass{article}

\usepackage{amsmath}
\usepackage{bm}

\title{Autoregressive Model as a State Space Model}
\author{Ivan Jacob Agaloos Pesigan}
\date{}

\begin{document}

\maketitle

\section{Measurement Model}

The measurement model is given by

\begin{equation}
    Y
    =
    \nu
    +
    \lambda
    \eta
    +
    \varepsilon
    \quad
    \mathrm{with}
    \quad
    \varepsilon
    \sim
    \mathcal{N}
    \left(
    0,
    \theta^{2}
    \right)
\end{equation}

\noindent where $Y$, $\eta$, and $\varepsilon$ are random variables and $\nu$, $\lambda$, and $\theta^{2}$ are model parameters. $Y$ is the observed random variable, $\eta$ is the latent random variable,
and $\varepsilon$ is the random measurement error while $\nu$ is the intercept, $\lambda$ is the factor loading, and $\theta^{2}$ is the variance of $\varepsilon$. For the autoregressive model some constraints are applied specifically $\nu = 0$, $\lambda = 1$,
$\varepsilon = 0$. Such that 

\begin{equation}
    Y = \eta .
\end{equation}

\subsection{Example}

The measurement model is given by

\begin{equation}
    \begin{array}{ccccccc}
         Y & = & \nu & + & \lambda \eta & + & \varepsilon \\
    Y
    & = &
    0
    & + &
    1
    \eta
    & + &
    0 \\
    Y
    & = &
    &   &
    \eta .
    &   &
    \\
    \end{array}
\end{equation}

\section{Dynamic Structure}

The dynamic structure is given by

\begin{equation}
    \eta_{t}
    =
    \alpha
    +
    \boldsymbol{\beta}
    \boldsymbol{\eta}_{l}
    +
    \zeta_{t}
    \quad
    \mathrm{with}
    \quad
    \zeta_{t}
    \sim
    \mathcal{N}
    \left(
    0,
    \psi^{2}
    \right)
\end{equation}

\noindent where $\eta_{t}$, $\boldsymbol{\eta}_{l}$, and $\zeta_{t}$ are random variables and $\alpha$, $\boldsymbol{\beta}$, and $\psi^{2}$ are model parameters.
$\eta_{t}$ is the latent variable at time $t$, $\boldsymbol{\eta}_{l}$ is a vector of latent variables at lags $l = \left\{ 1, \dots, p \right\}$, and $\zeta_{t}$ is the dynamic noise at time $t$ while 
$\alpha$ is the intercept, $\boldsymbol{\beta}$ is a row vector of autoregression coefficients, and $\psi^{2}$ is the variance of $\zeta_{t}$.

\subsection{Example}

The random variables and the parameters are given below for $p = 2$.

\begin{equation}
    \eta_{t}
\end{equation}

\begin{equation}
    \boldsymbol{\eta}_{l}
    =
    \left(
    \begin{array}{c}
        \eta_{t - 1} \\
        \eta_{t - 2} \\
    \end{array}
    \right)
\end{equation}

\begin{equation}
    \alpha
\end{equation}

\begin{equation}
    \beta
    =
    \left(
    \begin{array}{cc}
        \beta_{1} & \beta_{2} \\
    \end{array}
    \right)
\end{equation}

\begin{equation}
    \begin{array}{c|ccc|}
        \beta_{1} & \eta_{t} & \text{regressed on} & \eta_{t - 1} \\
        \beta_{2} & \eta_{t} & \text{regressed on} & \eta_{t - 2} \\
    \end{array}
\end{equation}

\begin{equation}
    \psi^{2}
\end{equation}

\section{Initial Condition}

In the state space model, the initial value of the latent variable $\eta$ given by $\eta_{\mid 0}$ need to be specified. We can sample from a particular distribution, for example, the normal distribution as follows

\begin{equation}
    \eta_{\mid 0} \sim \mathcal{N} \left( \mu_{0}, \sigma^{2}_{0} \right) .
\end{equation}

\end{document}

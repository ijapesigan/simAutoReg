\documentclass{article}

\usepackage{amsmath}
\usepackage{bm}

\title{Vector Autoregressive Model as a State Space Model}
\author{Ivan Jacob Agaloos Pesigan}
\date{}

\begin{document}

\maketitle

\section{Measurement Model}

The general form of the measurement model is given by

\begin{equation}
    \mathbf{y}
    =
    \boldsymbol{\nu}
    +
    \boldsymbol{\Lambda}
    \boldsymbol{\eta}
    +
    \boldsymbol{\varepsilon}
    \quad
    \mathrm{with}
    \quad
    \boldsymbol{\varepsilon}
    \sim
    \mathcal{N}
    \left(
    \mathbf{0},
    \boldsymbol{\Theta}
    \right)
\end{equation}

\noindent where $\mathbf{y}$, $\boldsymbol{\eta}$, and $\boldsymbol{\varepsilon}$ are random variables and $\boldsymbol{\nu}$, $\boldsymbol{\Lambda}$, and $\boldsymbol{\Theta}$ are model parameters. $\mathbf{y}$ is a vector of observed random variables, $\boldsymbol{\eta}$ is a vector of latent random variables,
and $\boldsymbol{\varepsilon}$ is a vector of random measurement errors while $\boldsymbol{\nu}$ is a vector of intercept, $\boldsymbol{\Lambda}$ is a matrix of factor loadings, and $\boldsymbol{\Theta}$ is the covariance matrix of $\boldsymbol{\varepsilon}$. For the vector autoregressive model some constraints are applied specifically $\boldsymbol{\nu} = \mathbf{0}$, $\boldsymbol{\Lambda} = \mathbf{I}$,
$\boldsymbol{\varepsilon} = \mathbf{0}$. Such that 

\begin{equation}
    \mathbf{y} = \boldsymbol{\eta} .
\end{equation}

\subsection{Three Variable Model Example}

Let $\mathbf{y} = \left\{ Y_{1} , Y_{2}, Y_{3} \right\}^{\prime}$. The measurement model is given by

\begin{equation}
    \begin{array}{ccccccc}
         \mathbf{y} & = & \boldsymbol{\nu} & + & \boldsymbol{\Lambda} \boldsymbol{\eta} & + & \boldsymbol{\varepsilon} \\
    \left(
    \begin{array}{c}
        Y_{1} \\
        Y_{2} \\
        Y_{3} \\
    \end{array}
    \right)
    & = &
    \left(
    \begin{array}{c}
        0 \\
        0 \\
        0 \\
    \end{array}
    \right)
    & + &
    \left(
    \begin{array}{ccc}
        1 & 0 & 0 \\
        0 & 1 & 0 \\
        0 & 0 & 1 \\
    \end{array}
    \right)
    \left(
    \begin{array}{c}
        \eta_{1} \\
        \eta_{2} \\
        \eta_{3} \\
    \end{array}
    \right)
    & + &
    \left(
    \begin{array}{c}
        0 \\
        0 \\
        0 \\
    \end{array}
    \right) \\
    \left(
    \begin{array}{c}
        Y_{1} \\
        Y_{2} \\
        Y_{3} \\
    \end{array}
    \right)
    & = &
    &   &
    \left(
    \begin{array}{c}
        \eta_{1} \\
        \eta_{2} \\
        \eta_{3} \\
    \end{array}
    \right) .
    &   &
    \\
    \end{array}
\end{equation}

\section{Dynamic Structure}

The general form of the dynamic structure is given by

\begin{equation}
    \boldsymbol{\eta}_{t}
    =
    \boldsymbol{\alpha}
    +
    \boldsymbol{\beta}
    \boldsymbol{\eta}_{l}
    +
    \boldsymbol{\zeta}_{t}
    \quad
    \mathrm{with}
    \quad
    \boldsymbol{\zeta}_{t}
    \sim
    \mathcal{N}
    \left(
    \mathbf{0},
    \boldsymbol{\Psi}
    \right)
\end{equation}

\noindent where $\boldsymbol{\eta}_{t}$, $\boldsymbol{\eta}_{l}$, and $\boldsymbol{\zeta}_{t}$ are random variables and $\boldsymbol{\alpha}$, $\boldsymbol{\beta}$, and $\boldsymbol{\Psi}$ are model parameters.
$\boldsymbol{\eta}_{t}$ is a vector of latent variables at time $t$, $\boldsymbol{\eta}_{l}$ is a vector of latent variables at lags $l = \left\{ 1, \dots, p \right\}$, and $\boldsymbol{\zeta}_{t}$ is a vector of dynamic noise at time $t$ while 
$\boldsymbol{\alpha}$ is a vector of intercepts, $\boldsymbol{\beta}$ is a matrix of autoregression and cross regression coefficients, and $\boldsymbol{\Psi}$ is the covariance matrix of $\boldsymbol{\zeta}_{t}$.

\subsection{Three Variable Model Example}

Let $\boldsymbol{\eta}_{t} = \left\{ \eta_{1_{t}}, \eta_{2_{t}}, \eta_{3_{t}} \right\}^{\prime}$. The random variables and the parameters are given below.

\begin{equation}
    \boldsymbol{\eta}_{t}
    =
    \left(
    \begin{array}{c}
        \eta_{1_{t}} \\
        \eta_{2_{t}} \\
        \eta_{3_{t}} \\
    \end{array}
    \right)
\end{equation}

\begin{equation}
    \boldsymbol{\eta}_{l}
    =
    \left(
    \begin{array}{c}
        \eta_{1_{t - 1}} \\
        \eta_{2_{t - 1}} \\
        \eta_{3_{t - 1}} \\
        \eta_{1_{t - 2}} \\
        \eta_{2_{t - 2}} \\
        \eta_{3_{t - 2}} \\
    \end{array}
    \right)
\end{equation}

\begin{equation}
    \boldsymbol{\alpha}
    =
    \left(
    \begin{array}{c}
        \alpha_{1} \\
        \alpha_{2} \\
        \alpha_{3} \\
    \end{array}
    \right)
\end{equation}

\begin{equation}
    \boldsymbol{\beta}
    =
    \left(
    \begin{array}{cccccc}
        \beta_{1} & \beta_{2} & 0 & 0 & 0 & \beta_{3} \\
        0           & \beta_{4} & \beta_{5} & 0 & 0 & 0 \\
        0           & 0 & \beta_{6} & 0 & 0 & 0 \\
    \end{array}
    \right)
\end{equation}

\begin{equation}
    \begin{array}{c|ccc|}
        \beta_{1} & \eta_{1_{t}} & \text{regressed on} & \eta_{1_{t - 1}} \\
        \beta_{2} & \eta_{1_{t}} & \text{regressed on} & \eta_{2_{t - 1}} \\
        \beta_{3} & \eta_{1_{t}} & \text{regressed on} & \eta_{3_{t - 2}} \\
        \beta_{4} & \eta_{2_{t}} & \text{regressed on} & \eta_{2_{t - 1}} \\
        \beta_{5} & \eta_{2_{t}} & \text{regressed on} & \eta_{3_{t - 1}} \\
        \beta_{6} & \eta_{3_{t}} & \text{regressed on} & \eta_{3_{t - 1}} \\
    \end{array}
\end{equation}

\begin{equation}
    \boldsymbol{\Psi}
    =
    \left(
    \begin{array}{ccc}
        \psi_{1, 1} & 0 & 0 \\
        0 & \psi_{2, 2} & 0 \\
        0 & 0 & \psi_{3, 3} \\
    \end{array}
    \right)
\end{equation}

\section{Initial Condition}

In the state space model, the initial values of the latent variables $\boldsymbol{\eta}$ given by $\boldsymbol{\eta}_{\mid 0}$ need to be specified. We can sample from a particular distribution, for example, the multivariate normal distribution as follows

\begin{equation}
    \boldsymbol{\eta}_{\mid 0} \sim \mathcal{N} \left( \boldsymbol{\mu}_{0}, \boldsymbol{\Sigma}_{0} \right)
\end{equation}

\noindent where $\boldsymbol{\mu}_{0}$ is a vector of means and $\boldsymbol{\Sigma}_{0}$ is a covariance matrix.

\subsection{Three Variable Model Example}

The mean vector and covariance matrix are given by

\begin{equation}
    \boldsymbol{\mu}_{0}
    =
    \left(
    \begin{array}{c}
        \mu_{0_{1}} \\
        \mu_{0_{2}} \\
        \mu_{0_{3}} \\
    \end{array}
    \right) ,
    \quad
    \text{and}
\end{equation}

\begin{equation}
    \boldsymbol{\Sigma}_{0}
    =
    \left(
    \begin{array}{ccc}
        \sigma_{0_{1, 1}} & \sigma_{0_{1, 2}} & \sigma_{0_{1, 3}} \\
        \sigma_{0_{2, 1}} & \sigma_{0_{2, 2}} & \sigma_{0_{2, 3}} \\
        \sigma_{0_{3, 1}} & \sigma_{0_{3, 2}} & \sigma_{0_{3, 3}} \\
    \end{array}
    \right) .
\end{equation}

\end{document}
